\documentclass[12pt,letterpaper]{article}
\begin{document}
\begin{titlepage}
\begin{center}
\title{MAKERERE UNIVERSITY \\COLLEGE OF COMPUTING AND INFORMATION SCIENCES\\ RESEARCH METHODOLOGY}
\author{Sendaula Brian}
\maketitle
\end{center}
\begin{flushleft}
\begin{center}
\title{NAME:	     SENDAULA BRIAN\\
STUDENTS NUMBER:	 214023415\\
REGISTRATION NUMBER: 14/U/24011/EVE}
\maketitle
\end{center}
\end{flushleft}
\end{titlepage}
  \begin{center}
  \line(1,0){300}\\
  [0.25in]
  \huge{\bfseries A REPORT ON MAKING A PAPER BAG}\\
  [2mm]
  \line(1,0){200}\\
  \maketitle
  \end{center}

\section{Introduction}
Making paper bags has become a new business trend for small scale amateur entrepreneurs in Uganda at the moment. It is a rewarding business for those that wish to make some extra income for their common expenditures. This report introduces the ‘how to’ of making your own paper bags.
According to wikihow.com, it can be noted that you can make your own paper bag with some old magazines newspaper, or simply craft paper lying around.   
\section{These are the steps to follow:}

Section one of two:
1.	Choose and gather your materials. This depends on the type of bag you want and whether or not it should have a handle. Set aside scissors, glue, ruler and a pencil. A thin piece of rope or ribbon works fine to create a handle. You can also gather some feathers, crayons or paint to decorate the bags.
2.	Step two: Cut a piece of paper to 9.5 by 15 inches. A ruler measures these out and a pencil can be used to stencil out the shape.

Section two of two: 

Assembling Your Paper Bag
Place the cut out paper in front of you on a flat surface. Make sure to place it in “landscape” orientation or long sides up and down, short sides to the left and right. If you have decorated your paper, make sure the decorations are dry and faced down. Fold the bottom edge of the paper up 2 inches (5 cm) and sharply crease the fold.  Locate the center points of the top and bottom edges. Maintaining a landscape orientation, bring the short sides together as though you were folding the whole thing in half, and pinch the top and bottom of the would-be fold to mark where the center of each long side is. Lightly mark these spots with a pencil.
Mark the paper again a half inch (13 mm) to both the left and right of each center point. When you’re done, you should have six marks total: three in the center of one long edge of your paper and three on the other.

    Fold the sides of the bag into place. Be sure to maintain the landscape orientation as you work to fold the sides. Flip the paper over, re-fold the left and right sides downward toward the center, and glue them where they overlap. Be sure to fold along the same lines as before. Let the glue dry completely before moving on to the next step.
    
    Flip the bag over so that it sits glued-side-down. Make sure to orient it so that one of the open ends points toward you.
    
    Fold the side-creases inward to create a slight accordion effect. You will be make the sides of the bag so that it opens up as a rectangle. With your ruler, measure inwards about 1.5 inches (3.8 cm) from the left-hand side of the bag. Lightly mark this with your pencil. Push the left side-crease of the bag inwards toward the interior of the bag. Do this until the left-hand mark you made in the previous step sits on the outer edge of where the paper is bending. When you’re done, the body of the bag should fold inwards on either side just like a grocery-shopping bag.

    Prepare the bottom of the bag. To determine which end is the bottom, look for the crease lines you folded previously that mark the bottom of the bag. Keep the bag flattened for now and prepare the bottom:
    
    Fold and glue the bottom of the bag into place. Once you've determined where the bottom of your bag is, piece together the bottom. Fold the bag 4 inches (10 cm) up from the bottom and crease it along this line. Keeping the rest of the bag flattened, prop open up the bottom of the bag. The inward-flaring creases should pop open, forming a square edge. Piece together the bottom of the bag. You will be folding a few sides to the center, using their triangular shape to ensure the bottom of the bag is evenly put together.
    
    Fold the left and right sides of the open, square-shaped bottom completely down. Use the outermost edge of each interior triangle as a guide. When you’re done, the bottom area should have 8 sides like an elongated octagon instead of 4 sides like it had before. Fold the bottom strip of the “octagon” upwards towards the center of the bottom of the bag. Fold the top strip of the “octagon” downwards towards the center of the bottom of the bag. The bottom should now be neatly folded shut; glue the edges where they overlap and let dry. Pop the bag open. Make sure the bottom is completely closed off and that there are no gaps in the glued edges. Add your handles. You can use a ribbon, rope, or string to make the handles or you can leave your bag as is without handles.
    
    Hold the two top of your bag closed and use a hole-puncher or pencil to make 2 holes at the top of your bag. Don't punch your holes too close to the edge of the bag or the weight of your bag plus anything inside it could break the handle. Reinforce the holes by lining the edges of the using clear tape or glue. Slide the ends of your handle string through the holes and make a knot on your handle string on the inside of the bag. Make sure the knot is big enough so it doesn't slide through the hole. You may have to tie another knot over the existing knot to increase its size. The knot keeps the handle in place and there you have a nice looking bag for use.

 \subsection{REFERENCE SOURCES:}
 www.auntannie.com/BoxesBags/EasyBag/
 Paper-design.wonderhowto.com/how-to/make-paper-bag-out-newspaper-335787/
\end{document}
